\documentclass[a4paper]{article}

%% Useful packages
\usepackage{amsmath}
\usepackage{graphicx}
\usepackage{amsmath}
\usepackage{amssymb}
\usepackage[colorinlistoftodos]{todonotes}
\usepackage[colorlinks=true, allcolors=blue]{hyperref}

\usepackage{outlines}
\usepackage{enumitem}
\setenumerate[1]{label=\Roman*.}
\setenumerate[2]{label=\Alph*.}
\setenumerate[3]{label=\roman*.}
\setenumerate[4]{label=\alph*.}

\title{Network sampling notes}
\author{}

\begin{document}
\maketitle

% \begin{abstract}
% Your abstract.
% \end{abstract}


\section{Paper outline}

\begin{outline}[enumerate]
\1 Problems to address
	\2 How should you sample an online network?
    \2 How close is link tracing to optimal?
    	\3 For cross group links?
        \3 For group sizes?
        \3 For node centrality?
	\2 How does this work in cases where nodes are imperfectly predicted?
\1 Previous work
	\2 Sampling methods
    \2 Homophily metrics
\1 Simplifying homophily metrics to what we actually measure
	\2 Group sizes
    \2 Crossgroup links
    \2 Degree
    \2 Others?
\1 Comparison of sampling methods for
	\2 Sampling frame vs not
    \2 Methods where no sampling frame is available
\1 Performance of various sampling methods for recovering measures
\1 Case where node attributes are inferred (e.g. have error)
\1 Empirical demonstration with Twitter and gender

\end{outline}

\section{Abstract}

Social scientists have long studied homophily between socially salient groups (e.g. race, ethnicity, gender, age), but such analysis is difficult online since frequently neither group labels nor sampling frames are known. We consider the case where a researcher wishes to recover homophily between groups in an online setting, starting from arbitrary seed nodes and without sampling the entire platform. Since machine learning classifiers are often used to predict group labels online, we consider the case where an imperfect classifier is used with a known error rate learned from test data. While many measures of homophily (assortativity, segregation) exist, the vast majority of measures use cross group link counts and node degree. We therefore study the ability of different sampling techniques to recover these two statistics, finding that using a link tracing design works best.

\section{Introduction}

Sociologists have studied interactions between socially salient groups for more than a half century \cite{coleman_relational_1958,katz_}. In recent years, the study of \emph{homophily}, or the tendency to observe similar individuals interacting at higher rates, has become an active interdisciplinary area of research in social science \cite{mcpherson_birds_2001}, treated as an important finding itself \cite{marsden_core_1987,lewis_tastes_2008}, an outcome of more microlevel processes \cite{currarini_economic_2009,bramoulle_homophily_2012}, and a cause of important network-level properties \cite{barabasi_emergence_1999} and global behavioral outcomes \cite{}. Homophily has been observed along many dimensions, including race, gender, education, and network position \cite{}.

Despite the importance of understanding homophily, studies of online interaction often draw on convenience samples and come without important demographic labels. Convenience samples, such as those obtained from sites like Twitter using the public search API, often provide useful instruments for studying behavior despite differing from the underlying population in unknown ways. In some studies, the lack of demographic labels is addressed through prediction methods relying on publicly available data \cite{} or crowdsourced labels and machine learning methods \cite{}.

Recently, studies have examined network sampling online in cases where there are multiple groups and a homophily-driven pattern of interaction \cite{wang_sampling_2015,karimi_visibility_2017,wagner_sampling_2017}. We build on these studies while incorporating insights from the respondent driven sampling literature \cite{heckathorn_respondent-driven_2002,salganik_sampling_2004,goel_respondent-driven_2009} to make two contributions. First, we show that link tracing designs inspired by RDS can be used to recover the statistics used to compute a wide variety of homophily measures. We provide natural baselines to determine the statistical efficiency of this approach, and assess its efficacy with simulation. We note that RDS can be applied directly if the research goal is to determine population proportions (e.g. fraction of women using a platform), although we are unaware of this technique being employed in online settings. Second, we show how using a classifier with known error rate impacts homophily estimates, and provide a method to correct this error when certain assumptions are made about the error.

These two insights together enable using link tracing designs to estimate group interaction rates while starting from arbitrary seed nodes and without ``sampling'' the entire population. While higher order network measures often require the entire network \cite{kossinets_effects_2006}, a great deal can be learned about homophily from a small sample of the data. We demonstrate this by studying gender homophily on Twitter, using a link tracing design of 10,000 Twitter users, which can be collected in a few hours with a single public Twitter API key.

\section{Homophily measures}

\section{Link tracing for estimating functions of vertices}

If we assume that a network is connected (strongly connected for directed graphs), we can apply the Markov Chain Monte Carlo (MCMC) method. This allows computing an unbiased estimate of an arbitrary function of the nodes in a graph. To see this, we follow the formalization in \cite{goel_respondent-driven_2009}.

$V$ is the population of nodes of size $N$ from which we sample. For two nodes $x, y \in V$, we call $K(x, y)$ the \emph{kernel} which gives the probability of transitioning between $x$ and $y$. $K(x, y) \ge 0$ and $\sum_{y \in V} K(x, y) = 1$.

In respondent driven sampling (RDS), $K(x, y)$ is the probability of one node recruiting another. For us, $K(x, y)$ is the probability of a crawl jumping from $x$ to $y$ when following a certain type of online signal (e.g. @mentions).

By assuming that the network is connected (strongly connected for directed graphs), we can compute the unique \emph{stationary distribution} of the chain $\pi: V \to \mathbb{R}$. This stationary distribution can be written

\begin{equation}
\sum_{x \in V} \pi(x) K(x, y) = \pi(y)
\end{equation}

If we denote $\pi$ as the entire stationary vector and $K$ as the $V \times V$ row-stochastic matrix of transition probabilities, then this can be written compactly for all nodes as $\pi^T K = \pi$. From this, it's clear that if we know $K$ we can find $\pi$ since it is an eigenvector of $K$. However, we do not know $K$ (we usually do not know $V$, either). However, if our chain is in equilibrium ($X_0 \sim \pi$) a random walk samples from $\pi$.

We wish to obtain an unbiased estimate of the expectation of a function $f: V \to \mathbb{R}$

\begin{equation}
\mathbb{E}_\pi f = \sum_{i = 0}^{n-1} f(X_i) \pi(X_i)
\end{equation}

\noindent
which the MCMC method allows us to do with

\begin{equation}
\hat{\mathbb{E}}_\pi f = \frac{1}{n} \sum_{i = 0}^{n-1} f(X_i)
\end{equation}

\noindent
where $X_i$ indicates the $i$th realization of random walk $X$ with length $n$. RDS chooses $f$ to compute a \emph{uniform mean} over the nodes in order to obtain the proportion of individuals belonging to some group of interest.

\section{Simulation setup}

\section{Empirical analysis}

\section{Conclusion}


Often, no population sampling frame is available, necessitating the use of respondent-driven-sampling (RDS).


A fundamental problem for such studies is obtaining a representative sample of links in order to infer intergroup interaction rates. Internet-based methods often make it

Studies of large social networks have proliferated in recent years. Data is often gathered for such studies by

Network sampling methods are increasingly useful as the Internet makes social networks available for link tracing. We start from seed nodes and utilize a snowball-like sample to recover a statistic of interest. Such methods have focused on sampling population proportions \cite{heckathorn_respondent-driven_2002} or functions of node degree \cite{karimi_visibility_2017}. In this setting, sampling \emph{homophily} \cite{currarini_economic_2009} has not been studied before, despite its wide importance for social phenomena. We tackle this problem by focusing on the components of homophily: group sizes and within-group interaction rates. By decomposing the sampling problem into these two components, we can evaluate how sampling methods recover each individually, as well as how the error aggregates. Using simulated graphs, we study a variety of sampling methodologies and find (RESULTS).

\section{Notes}

\subsection{Project goals}

We have access to massive amounts of online networked data. From these data, we wish to estimate the rates of interaction (homophily) between socially salient groups.

Two main objectives of this project:

\begin{enumerate}
\item How does homophily in the sample relate to the social media platform?
\item How does homophily in the sample relate to the world?
\end{enumerate}

Answering 1 is doable. Answering 2 takes us into the realm of speculation and informed conjecture. While this

\subsection{Homophily calculation}

We follow the formalization in \cite{currarini_economic_2009}.

For a focal group $i$, call $s_i$ the average number of links to the ingroup and $d_i$ the average number of links to all outgroups.

Then we get a basic homophily measure

\begin{equation}
H_i = \frac{s_i}{d_i + s_i}
\end{equation}

This tells us nothing about random baselines, however. Group $i$ may be large or small. We denote $w_i$ the fraction of the population that belongs to $i$. If $H_i$ is higher than this baseline, we say that $i$ experiences \emph{inbreeding homophily}. We then normalize the statistic $H_i - w_i$ by the maximum possible bias, $1 - w_i$, which represents the case where there is perfect homophily (all ties ingroup minus the group size).

\begin{equation}
IH_i = \frac{H_i - w_i}{1 - w_i}
\end{equation}

\subsection{RDS framework}

We follow \cite{Salganik2004} and use the following notation. Assume that we have a graph with some nodes and edges. Further, assume all links are bidirectional. Then:

\begin{itemize}
\item $A$ and $B$ are socially relevant groups, such as race/ethnicity, gender, or educational attainment
\item $T_{AB}$: the number of links from $A$ to $B$
\item $C_{A,B}$: the probability an edge in $A$ leads to $B$
\item $R_A$: the total degree of $A$
\item $N_A$: the number of nodes in $A$
\item $D_A$: the average degree of $A$, $R_A / N_A$
\end{itemize}

Then we get the fundamental equality:

\begin{equation}
N_A D_A C_{A,B} = N_B D_B C_{B,A}
\end{equation}

This is true by definition assuming links are bidirected. It also gives us information about which information we need to infer the other parts. For instance, knowing any two of the following will let us infer the third: 1) number of nodes in each group; 2) the average degree of each group; 3) the probability of a crosslink.



\subsection{Random assignment of node characteristics}

\subsection{Incorporating homophily on characteristics}


\section{The effect of random node misclassification on homophily}

terms

\begin{enumerate}
\item $n_a$, $n_b$: number in $a$ and $b$
\item $N$: population size
\item $d_a$, $d_b$: number of links from a member of $a$, $b$
\item $d_{ab}$, $d_{ba}$: number of crossgroup links for each group, $a\to b$, $b\to a$
\item $d_a/n_a$, $d_b/n_b$: number of links per node
\item $c_{ab}$, $c_{ba}$: probability of crossgroup link from $a\to b$ and $b\to a$
\end{enumerate}

in expectation, there are $n_a * d_a * c_{ab}$ links from $a\to b$

misclassification messes up a few things:

1. estimates of population size
2. within-between link counts for each group
3. cross group link probability

say we mistake a random $a$ node for a $b$ node. this changes the group proportion estimates by $n_a/N - (n_a - 1)/N$ and $n_b/N + (n_b + 1)/N$.

it changes the crossgroup link probability by taking a random $a$ node and making its $a\to a$ links $b\to a$ and its $a\to b$ links $b\to b$. so we add the expected number of $d_{aa}$ links per node to $b\to a$ and the expected number of $d_{ab}$ links per node to $b\to b$

note that we can write $c_{ab} = d_{ab} / d_a$

then we get expressions for the updated $c_{aa}$, $c_{ab}$, $c_{ba}$, $c_{bb}$

\begin{enumerate}
\item $c_{aa} = (d_{aa} - E[d_{aa}/n_a]) / (d_a - E[d_{aa}/n_a])$ subtract expected $aa$ links
\item $c_{ab} = (d_{ab} - E[d_{ab}/n_a]) / (d_a - E[d_{ab}/n_a])$ subtract expected $ab$ links
\item $c_{ba} = (d_{ba} + E[d_{aa}/n_a]) / (d_b + E[d_{aa}/n_a])$ add expected $aa$ links
\item $c_{bb} = (d_{bb} + E[d_{ab}/n_a]) / (d_b + E[d_{ab}/n_a])$ add expected $ab$ links
\end{enumerate}


so we can see the expected effects of moving a random node from $a$ to $b$. since $d_a > d_{aa}$ (except in case of perfect homophily), subtracting the same number from the numerator and denominator of $d_{aa} / d_a$ will lead to an *underestimate* estimate of the true value. e.g. consider $(d_{aa} - x) / (d_a - x)$.

likewise we should get an over-estimate for $c_{ba}$, $c_{bb}$ since we basically have $(d_{ab} + x) / (d_b + x)$ and $(d_{bb} + x) / (d_b + x)$.

this is true for moving one node, but i think this argument fails for switching lots of nodes. for instance if we randomly flipped 20\% of the nodes in the graph, in expectation according to this argument we should not change homophily. but we clearly will by making the overall state of the graph "more random" and less ordered by homophily. the reason is that after switching lots of nodes the $d_{ab}$ and $d_{ba}$ numbers are substantially closer to random mixing than they were at the start.

how do we quanitify this?

we could make a dyadic independence assumption and see how well that works IRL

we really want like an estimate of how close to randomly mixed the links are after a certain number of random flips

Assume that each node has an equal probability p of being flipped. For each edge and node type, we can write out the possible transitions and their probabilities. For edges:

\begin{itemize}
\item The probability that the edge will remain as-is is $(1-p)^2$
\item The probability that only the first value of the edge will change is $p(1-p)$
\item The probability that only the second value of the edge will change is $p(1-p)$
\item The probability that both values of the edge will change is $p^2$
\end{itemize}

The total odds for any starting edge state are $(1-p)^2 + 2p(1-p) + p^2 = 1$.

And for nodes, the probability that the node will remain as is is $1-p$ and the probability that it will flip is $p$, by definition.

Then for each final edge- or node-state, the probability of that state is the sum of the products of the probabilities of all the initial states with their transition probability to the final state in question. For example, the probability that a link in the final network is from "a" to "a" is:

\begin{equation*}
c_{aa}' = (1-p)^2c_{aa} + p(1-p)c_{ab} + p(1-p)c_{ba} + p^2c_{bb}
\end{equation*}

And the probability that a node in the final network is "a" is:
\begin{equation*}
w_a' = (1-p)w_a + pw_b
\end{equation*}

The full systems of linear equations can be expressed in the form of a linear equation
\begin{equation*}
\vec{d}' = M\vec{d}
\end{equation*}
where for nodes we have
\begin{equation*}
\vec{d} = \begin{pmatrix}w_a\\w_b\end{pmatrix}
\end{equation*}
and
\begin{equation}
M = \begin{bmatrix}
1-p & p\\
p & 1-p
\end{bmatrix}
\end{equation}
and for edges we have
\begin{equation*}
\vec{d} = \begin{pmatrix}c_{aa}\\c_{ab}\\c_{ba}\\c_{bb}\end{pmatrix}
\end{equation*}
and
\begin{equation} M =
\begin{bmatrix}
(1-p)^2 & p(1-p) & p(1-p) & p^2\\
p(1-p) & (1-p)^2 & p^2 & p(1-p)\\
p(1-p) & p^2 & (1-p)^2 & p(1-p)\\
p^2 & p(1-p) & p(1-p) & (1-p)^2
\end{bmatrix}
\end{equation}

From this information, the desired metrics such as homophily can be calculated.

For different $p_a$ and $p_b$
\begin{equation}M =
\begin{bmatrix}
(1-p_a)^2 & p_b(1-p_a) & p_b(1-p_a) & p_bp_b\\
p_a(1-p_a) & (1-p_a)(1-p_b) & p_ap_b & p_b(1-p_b)\\
p_a(1-p_a) & p_ap_b & (1-p_a)(1-p_b) & p_b(1-p_b)\\
p_a^2 & p_a(1-p_b) & (1-p_b)p_a & (1-p_b)^2
\end{bmatrix}
\end{equation}
for edges and
\begin{equation}
M = \begin{bmatrix}
1-p_a & p_b\\
p_a & 1-p_b
\end{bmatrix}
\end{equation}
for nodes.

If you want to find the expected true edge and node counts after misclassification, then you can just invert this system.
\bibliographystyle{abbrv}
\bibliography{scibib}

\end{document}
