\documentclass[a4paper]{article}

%% Useful packages
\usepackage{amsmath}
\usepackage{graphicx}
\usepackage{amsmath}
\usepackage{amssymb}
\usepackage{verbatim}
\usepackage[colorinlistoftodos]{todonotes}
\usepackage[colorlinks=true, allcolors=blue]{hyperref}

\usepackage{outlines}
\usepackage{enumitem}
\setenumerate[1]{label=\Roman*.}
\setenumerate[2]{label=\Alph*.}
\setenumerate[3]{label=\roman*.}
\setenumerate[4]{label=\alph*.}

\title{Network sampling notes}
\author{}

\begin{document}
\maketitle

% \begin{abstract}
% Your abstract.
% \end{abstract}


\section{Paper outline}

\begin{outline}[enumerate]
\1 Problems to address
	\2 Sampling statistics of graph nodes
    \2 Sampling statistics of graph nodes with an imperfect classifier
    \2 How close is our proposed method to optimal?
    \2 How can we see if this works in real life?
\1 Previous work
	\2 Naive methods (snowball, pure RW, edge, node, ego network)
    \2 RDS
    	\3 RDS as MCMC intro
        \3 RDS tweaks
    	\3 Importance resampling
	\2 Homophily
    	\3 Areas where it exists
        \3 Various indexes
        \3 Coleman index
    \2 Visibility
    	\3 What is it?
    \2 (What's a classifier?)
\1 Findings (each with perfect/imperfect classification)
    \2 Graph generation
    	\3 Power law with homophily
	\2 Group size
    \2 Crossgroup links
    \2 Coleman homophily
    \2 Visibility of minority group
    \2 Crossnational ties
\end{outline}

\section{Abstract}

Homophily, or the tendency of similar people to interact, is a consistent empirical finding in networks research. Despite this consistency, it is still difficult and costly to obtain precise estimates of homophily in networks. We address the problem of obtaining an unbiased estimate of homophily between groups in online social networks when 1) a population sampling frame is not known; and 2) group membership is predicted using an imperfect classifier. We show that using a random walk design and a correction for classifier error leads to an unbiased homophily estimate. Our procedure applies straightforwardly to other statistics, such as the visibility of minority group nodes and the size of the minority group. We demonstrate the usefulness of this technique by estimating the fraction of Twitter users which reside in the U.S. and the homophily of U.S./non-U.S. Twitter users.

\section{Introduction}

Sociologists have studied interactions between socially salient groups for more than a half century \cite{coleman_relational_1958}. In recent years, the study of \emph{homophily}, or the tendency to observe similar individuals interacting at higher rates, has become an active interdisciplinary area of research in social science \cite{mcpherson_birds_2001}, treated as an important finding itself \cite{marsden_core_1987,lewis_tastes_2008}, an outcome of more microlevel processes \cite{currarini_economic_2009,bramoulle_homophily_2012}, and a cause of important network-level properties \cite{barabasi_emergence_1999} and global behavioral outcomes \cite{}. Homophily has been observed along many important dimensions, including race, gender, education, and network position \cite{}.

Estimating homophily precisely requires a statistically valid sample of both nodes and edges in a network. Obtaining such a sample when no sampling frame is known requires using a reweighted random walk technique called respondent driven sampling (RDS) \cite{heckathorn_respondent-driven_2002}. RDS has been shown to successfully sample offline hard-to-reach social populations such as drug users, yet is seldom used in an online context despite strong performance \cite{ribeiro_estimating_2010}.

Used appropriately, RDS can provide an unbiased sample of homophily in online networks. This occurs because the random walk part of RDS performs a random sample of the edges in the graph, while the RDS procedure was developed to estimate population proportions of each group (node sample).

Using a straightforward correction, we extend this result to the common case in online networks where membership in important social groups is inferred via a classifier with a known error rate (e.g. \cite{barbera_less_2016}).

While our main practical focus is homophily, we apply the result of previous RDS research \cite{goel_respondent-driven_2009} that RDS can provide an unbiased estimate of \emph{any} function of the nodes in a graph. In particular, we address estimating the visibility of minority group members \cite{karimi_visibility_2017}, showing that it is possible to obtain an unbiased estimate of the visibility of minority group members by applying RDS plus error correction.

Finally, since our technique is meant to be applied empirically, we perform a random walk crawl of Twitter users in order to estimate the fraction of Twitter users in the U.S. vs abroad, the level of homophily between them, and the fraction of non-U.S. users in the top 20\% of the degree distribution (visibility).

The paper is organized as follows: section <<1>> reviews the relevant literature for this paper, section <<2>> describes our simulations and provides results for the case with no classification error, section <<3>> shows the effects of classification error and the result of our correction technique, and section <<4>> presents the empirical analysis.

\section{Previous work}

This work draws on several lines of research in different fields. We provide a summary of each line of work here. Readers familiar with the content of a section may feel free to skip ahead.

\subsection{Sampling in graphs}

The main technique we draw on here, respondent driven sampling (RDS), was motivated by the desire to validly sample the size of a group (e.g. the proportion of males) in a given network. In particular, RDS assumes that the researcher does not have a list of nodes and edges prior to sampling (no sampling frame), and must therefore learn about the graph through walking the graph only. RDS assumes that the seed nodes are chosen proportional to degree, that the graph is undirected and connected, and that the walker jumps randomly between the nodes.

From this, it is possible to reweight the walk to correctly recover group proportions, or to recover any function of the nodes. We develop this below.

RDS is often contrasted to snowball sampling, or the procedure of starting with an arbitrary node and ``snowballing'' out by collecting information on the seed node's neighbors, all of the neighbors of the seed node's neighbors, and so forth.

Other types of sampling are imaginable if we have a sampling frame \cite{karimi_visibility_2017}: sampling the nodes randomly, sampling edges randomly, and sampling ego networks randomly (as done in the General Social Survey).

Throughout, we contrast these five sampling techniques: 1) RDS; 2) snowball sampling; 3) node sampling; 4) edge sampling; 5) ego network sampling. This gives an idea of the performance and trade-offs of each.

\subsubsection{RDS as MCMC}

In this section, we develop the idea of respondent driven sampling (RDS) as Markov Chain Monte Carlo (MCMC), closely following \cite{goel_respondent-driven_2009}. We refer readers to \cite{heckathorn_respondent-driven_2002,salganik_sampling_2004,goel_respondent-driven_2009} (in that order) for an introduction to respondent driven sampling.

We assume there is a graph $G$ with nodes $V$ and edges $E$. The number of nodes is $N = |V|$. $E$ is undirected, meaning that $(v_1, v_2) \in E \implies (v_2, v_1) \in E$. For simplicity we assume that all edges have the same weight (probability to jump), although this is not required if weights are known. For the purposes of a Markov chain, $V$ are the states the chain can take on. We refer to the chain as $\chi$ with realized steps denoted $X_1, X_2, ...$.

For two nodes $x, y \in V$, we call $K(x, y)$ the \emph{kernel} which gives the probability of transitioning between $x$ and $y$. $K(x, y) \ge 0$ and $\sum_{y \in V} K(x, y) = 1$.

In offline RDS, $K(x, y)$ is the probability of one node recruiting another. For us, $K(x, y)$ is the probability of a crawl jumping from $x$ to $y$ when following a certain type of online signal (e.g. friend relationships).

By assuming that the network is connected, undirected, and has at least one triangle, we can compute the unique \emph{stationary distribution} of the chain $\pi: V \to \mathbb{R}$. This stationary distribution can be written

\begin{equation}
\sum_{x \in V} \pi(x) K(x, y) = \pi(y)
\end{equation}

If we denote $\pi$ as the entire stationary vector and $K$ as the $V \times V$ row-stochastic matrix of transition probabilities, then this can be written compactly for all nodes as $\pi^T K = \pi$. From this, it's clear $\pi$ is an eigenvector of $K$. In fact, under the assumptions we have made here the steady state probability can be computed directly. Let $d_x$ be the degree of $x$. Then $\pi(x) = d_x / \sum_y d_y$.

If our chain starts in equilibrium ($X_0 \sim \pi$) a random walk samples from $\pi$. If our chain starts out of equilibrium, then the chain ``bounces around'' before beginning to draw from the steady state distribution. The time to converge from the initial distribution to $\pi$ depends on many factors \cite{}.

The ultimate goal is to draw a sample where each node is weighted equally, but the chain samples nodes with probabilities given by $\pi$. Note that sampling from $\pi$ allows us to take the expectation of $f: V \to \mathbb{R}$ over $\pi$, denoted

\begin{equation}
\mathbb{E}_\pi f = \sum_{i = 0}^{n-1} f(X_i) \pi(X_i)
\end{equation}

\noindent
which is estimated by

\begin{equation}
\hat{\mathbb{E}}_\pi f = \frac{1}{n} \sum_{i = 0}^{n-1} f(X_i)
\end{equation}

\noindent
where $n$ is the length of the walk. This method takes the expectation of $f$ weighted by the probability $\pi(x)$ of being in each state $x$. In the case of a social network, links with high degree will be visited more often and therefore contribute more to the expectation.

Intuitively, we'd like to downweight the high degree nodes and upweight the low degree nodes so that we take an expectation with each node weighted equally, rather than over $\pi$. We can do this by

\begin{align}
E_\pi (\frac{f(X_i)}{N \cdot \pi(X_i)}) &= \sum_{i=0}^{N} \frac{f(X_i)}{N \cdot \pi(X_i)} \pi(X_i)\\
&= \frac{1}{N} \sum_{i=0}^{N} f(X_i).
\end{align}

This is an average over the nodes in the graph. By substituting $\frac{f(X_i)}{N \cdot \pi(X_i)}$ into Equation 4, we obtain

\begin{equation}
\frac{1}{n \cdot N} \sum_{i=0}^{n-1} \frac{f(X_i)}{\pi(X_i)}
\end{equation}

\noindent
which leads to the following estimator

\begin{align}
\hat{\mu} &= \frac{1}{\sum_{i=0}^{n-1} 1 / \pi(X_i)} \sum_{i=0}^{n-1} \frac{f(X_i)}{\pi(X_i)} \\
&=\frac{1}{\sum_{i=0}^{n-1} 1 / d_i} \sum_{i=0}^{n-1} \frac{f(X_i)}{d_i}.
\end{align}

Equation 9 can be computed with degrees $d_i$ from the sampled nodes only. This follows by replacing $\pi(X_i)$ with $d_i / \sum_x d_x$ and noting that the $\sum_x d_x$ terms cancel.

This estimator in Equation 9 indicates that a random walk on an undirected graph can sample a function $f$ of the nodes if the node weighted degrees are known up to a proportional constant.

Note that Equation 7 resembles the importance sampling estimator

\begin{equation}
\frac{1}{n} \sum_{i=0}^{n-1} \frac{f(X_i) p(X_i)}{q(X_i)}, X_i \sim q
\end{equation}

\noindent
if we say that $q(X_i) = \pi(X_k)$ and $p(X_i) = \frac{1}{N}$. In other words, we sample from $q$ but really want to take draws from $p$.

\subsection{Homophily measures}

Many measures of homophily have been proposed by network researchers \cite{bojanowski_measuring_2014}. Similar concepts are often called ``assortativity'' \cite{} or ``segregation'' \cite{}. While these methods have important differences, the common goal is to quantify the amount of nonrandom interaction between groups in networks.

One of the most widely known measures is Coleman's homophily index \cite{coleman_relational_1958}. This measure treats random mixing as its baseline, with 1 indicating perfect homophily and -1 indicating perfect heterophily. We will present results mainly using this measure, with a discussion of how our findings relate to other important homophily, assortativity, and segregation measures. We assume an undirected graph here.

To understand the Coleman homophily index, denoted $H^{Coleman}$, consider a set of social groups $G = \{A, B\}$. When groups are variables we will use the letters $x, y$. These groups have some number of members, $N_A$ and $N_B$, with the population size denoted as $N$. In the two-group case, $N = N_A + N_B$. There are also counts of ties in the network, with the number of ties from $A$ to $B$ called $m_{AB}$ and the fraction of ties called $\eta_{AB} = \frac{m_{AB}}{m_{AA} + m_{AB}}$.

The Coleman homophily measure chooses random tie formation as a baseline. In other words, if the proportion of nodes in group $A$ is $p_A = N_A / N$, then there is no homophily for group $A$ exactly when $\eta_{AA} = p_A$. For group $A$, homophily is

\begin{equation}
H_A^{Coleman} = \left\{\begin{array}{lr}
        \frac{\eta_{AA} - p_A}{1 - p_A} & \text{for } \eta_{AA} - p_A \ge 0\\
        \frac{\eta_{AA} - p_A}{p_A} & \text{for } \eta_{AA} - p_A < 0
        \end{array}\right\}
\end{equation}

This index is normalized so that it has its maximum at 1 and minimum at -1. Zero indicates no homophily for group $A$.

% We can extend this measure to the graph level by considering the total fraction of ties in the graph within the same group compared to the expected number of same-group ties given by chance. Denote $\eta_{ingroup} = \sum_x \eta_{xx}$ and

This measure requires two pieces of information up to constants of proportionality: 1) the number of nodes in each group $N_x$; 2) counts of links within and between groups $m_{xy}$. Other commonly used homophily measures can in principle be computed with these data, as seen in Table \ref{table1}. While the of this paper is to sample homophily, we can decompose this task into two separate ones: estimating group sizes and estimating link counts. Focusing on these lower-level measures clarifies where error can enter estimates of homophily, and also extends the analysis here to other measures which can be computed with the same data.

\begin{table}[!htbp] \centering
  \label{table1}
\begin{tabular}{@{\extracolsep{5pt}}lcc}
\\[-1.8ex]\hline
\hline \\[-1.8ex]
Name & $m_{xy}$ & $N_x$ \\
\hline \\[-1.8ex]
Coleman's homophily index & \checkmark & \checkmark \\
Freeman's segregation index & \checkmark & \checkmark \\
E-I index & \checkmark & \\
Assortativity & \checkmark & \\
Odds ratio within groups & \checkmark & \\
\hline
\hline \\[-1.8ex]
\end{tabular}
  \caption{Homophily measures and required data. A checkmark indicates that the measure is required to compute the index. $m_{xy}$ indicates that link counts are required for all groups $x, y \in G$ up to a constant of proportionality. $N_x$ indicates that node counts are required for all groups $x \in G$ up to a constant of proportionality.}
\end{table}

\subsection{Visibility}

The concept of visibility has been recently studied in the network literature \cite{karimi_visibility_2017, wagner_sampling_2017}. This concept is defined as the presence of members of a minority group in the top quantile of the degree distribution. In the case of sampling, we attempt to recover the proportion of the minority group in the top  quantile (e.g. the top fifth).


\subsection{Classification}

\section{Findings}

\subsection{Graph generation}

\subsection{Sampling group size}

\subsection{Sampling crossgroup links}

\subsection{Sampling Coleman homophily}

\subsection{Sampling minority group visibility}

\subsection{Empirical sampling of crossnational ties}


\newpage

Despite the importance of understanding homophily, studies of online interaction often draw on convenience samples and come without important demographic labels. Convenience samples, such as those obtained from sites like Twitter using the public search API, often provide useful instruments for studying behavior despite differing from the underlying population in unknown ways. In some studies, the lack of demographic labels is addressed through prediction methods relying on publicly available data \cite{} or crowdsourced labels and machine learning methods \cite{}.

Recently, studies have examined network sampling online in cases where there are multiple groups and a homophily-driven pattern of interaction \cite{wang_sampling_2015,karimi_visibility_2017,wagner_sampling_2017}. We build on these studies while incorporating insights from the respondent driven sampling literature \cite{heckathorn_respondent-driven_2002,salganik_sampling_2004,goel_respondent-driven_2009} to make two contributions. First, we show that link tracing designs inspired by RDS can be used to recover the statistics used to compute a wide variety of homophily measures. We provide natural baselines to determine the statistical efficiency of this approach, and assess its efficacy with simulation. We note that RDS can be applied directly if the research goal is to determine population proportions (e.g. fraction of women using a platform), although we are unaware of this technique being employed in online settings. Second, we show how using a classifier with known error rate impacts homophily estimates, and provide a method to correct this error when certain assumptions are made about the error.

These two insights together enable using link tracing designs to estimate group interaction rates while starting from arbitrary seed nodes and without ``sampling'' the entire population. While higher order network measures often require the entire network \cite{kossinets_effects_2006}, a great deal can be learned about homophily from a small sample of the data. We demonstrate this by studying gender homophily on Twitter, using a link tracing design of 10,000 Twitter users, which can be collected in a few hours with a single public Twitter API key.

\section{Homophily measures}


\section{Sampling on graphs}

The difficulty of sampling from online networks with homophily and multiple groups has recently drawn attention \cite{wagner_sampling_2017} in connection with the visibility minority group nodes \cite{karimi_visibility_2017}. Particularly in online settings, extreme degree disparities can cause a ``majority illusion'' where a huge fraction of users observe globally rare states in their ego networks (e.g. celebrity status) \cite{lerman_majority_2016}.

Many studies address sampling network centrality measures ... \cite{}

Previous work has considered five sampling methods: edge (E), node (N), snowball (S), random walk (RW), and reweighted random walk also known as respondent driven sampling (RDS) \cite{wagner_sampling_2017,wang_sampling_2015,heckathorn_respondent-driven_2002}. We argue that RDS offers a methodology to sample important properties of graphs including homophily.

% probably need notation for the chain itself, a fancy X?



\subsection{Variance}

\subsection{Distributions}

Up until this point we have focused on estimating the mean of a function $f$ over the population of nodes $V$. We may also be interested in the distribution of $f$, such as in the case of estimating the degree distribution of the graph.

When we draw samples from a random walk, we sample proportional to node degree. In other words, higher degree nodes have higher probabilities of being sampled. If we downweight high degree nodes and upweight low degree nodes, we can appropriately correct our sample to match the underlying degree distribution.

For instance, if we know node $i$ with degree $d_i$ is sampled with probability $\pi(X_i)$ and we wanted to sample it with probability $1/N$, then we can use the ratio $p/q$ to get the importance weight. In our case, this comes out to

\begin{equation}
\frac{p(X_i)}{q(X_i)} = \frac{1}{N \pi(X_i)} = \frac{D}{N d_i} = \frac{\bar{d}}{d_i}.
\end{equation}

After drawing a sample from the random walk, we can approximate the distribution of $f(X_i)$ using the importance weight to correct the distribution of $f$ obtained from the sample. This follows straightforwardly from the fact that weighting by $\frac{\bar{d}}{d_i}$ gives

\begin{equation}
P_{reweighted}(X_i) = \pi(X_i) \frac{\bar{d}}{d_i} = \frac{d_i}{D} \frac{D}{N d_i} = \frac{1}{N}.
\end{equation}

\noindent
Using the importance weight $\frac{\bar{d}}{d_i}$ to approximate samples drawn from $p$ is called importance resampling. In practice, we use the reweighted counterpart $\frac{\bar{d}}{d_i} / \sum_{j \in \chi} \frac{\bar{d}}{d_j}$.

Since our importance resample approximates draws from $p \sim \frac{1}{N}$, the distribution of $f$ on the resample will approximate the true distribution of $f$ in the graph $V$.

\begin{comment}

To see this, note that the true density of $f$ is given by $P(f(V) = k) = V_k / N$, where $f(V)$ denotes the distribution of $f$ in the graph, $V_k = |\{i \in V: f(X_i) = k\}|$ is the number of nodes which have value $f(X_i) = k$, and $N$ is the number of nodes in the graph.

Note that the random walk gives us an observation $f(X_i)$ with probability $\pi(X_i) = d_i / D$, where $D$ is the degree of the graph. If we multiply this probability by the ``inclusion weight'' in Equation 11, we obtain



\noindent
or the uniform distribution on nodes. Using this weight, we can see that in a sample of size $n$

\begin{equation}
\hat{P}(f(X_i) = k) = \frac{V_k(n)}{n} = \frac{}
\end{equation}

This implies that when we obtain a sample from the walk $\chi$, we can approximate the true distribution of $f$ by weighting the sampled elements by $\frac{\bar{d}}{d_i}$.

Practically speaking, we can bootstrap an estimate of the distribution $f(X_i)$. For each $i \in \chi$, we use $\frac{\bar{d}}{d_i}$ as the importance weight. We can see that this process converges in distribution to $f(V)$, or the distribution in the graph.

Assume $f(V)$ is a discrete distribution (e.g. degree). Then we need to show that $P(f(X_i) = k) \to P(f(V) = k)$ as $i \to \infty$, where $f$ indicates the true distribution and $k$ is some value that $f$ takes on. Denote the set of nodes with $f(X_i = k)$ as $V_k$. The probability that $f(X_i)$ takes on value $k$ in the walk is denoted $\pi(f(X_i) = k) = \sum_{V_k} d_i / D$, or the sum degree of the set of nodes with value $f(X_i) = k$, divided by the total degree of the graph. Assume our samples are from a chain in equilibrium, then the corrected distribution has

\begin{align}
P(f(X_i) = k) &= \pi(f(X_i) = k) \frac{\bar{d}}{d_i}\\
&= \frac{\sum_{V_k} d_j}{D} \frac{\bar{d}}{d_i}\\
&= \frac{\sum_{V_k} d_j}{D} \frac{D}{N d_i}\\
&= \frac{|V_k|}{N}\\
&= P(f(V) = k).
\end{align}

This argument indicates that weighting draws from the random walk by $\frac{\bar{d}}{d_i}$ faithfully replicates the distribution of $f(X_i)$. In practice, we estimate the mean degree from a walk, potentially the same one we bootstrap from. This introduces error according to the sampling variance of estimating the mean.

\end{comment}

\section{Computational experiments}

We follow recent work on preferential attachment networks with group homophily \cite{}. Simulating the creation of a standard powerlaw network, each node gets a fixed budget of links and chooses endpoints proportional to the degree of the receiving node \cite{}. In the case with groups and homophilious attachment on groups, both factors influence a node's decision to choose a destination for its link.

The framework we developed above allows us to restrict most of our thinking to the choice of $f$. When $f$ is node degree, we estimate graph mean degree; when $f$ is an indicator variable for a group, we estimate the size of that group.

\subsection{Homophily}

Coleman's homophily index requires an estimate of group sizes, and an estimate of cross-group interaction rates.

To get an estimate of group sizes, choose $f = $ in group.

\subsection{Degree distribution}


\subsection{Visibility}



\section{Empirical analysis}

\subsection{Restrictions}

\section{Conclusion}


Often, no population sampling frame is available, necessitating the use of respondent-driven-sampling (RDS).


A fundamental problem for such studies is obtaining a representative sample of links in order to infer intergroup interaction rates. Internet-based methods often make it

Studies of large social networks have proliferated in recent years. Data is often gathered for such studies by

Network sampling methods are increasingly useful as the Internet makes social networks available for link tracing. We start from seed nodes and utilize a snowball-like sample to recover a statistic of interest. Such methods have focused on sampling population proportions \cite{heckathorn_respondent-driven_2002} or functions of node degree \cite{karimi_visibility_2017}. In this setting, sampling \emph{homophily} \cite{currarini_economic_2009} has not been studied before, despite its wide importance for social phenomena. We tackle this problem by focusing on the components of homophily: group sizes and within-group interaction rates. By decomposing the sampling problem into these two components, we can evaluate how sampling methods recover each individually, as well as how the error aggregates. Using simulated graphs, we study a variety of sampling methodologies and find (RESULTS).

\section{Notes}

\subsection{Project goals}

We have access to massive amounts of online networked data. From these data, we wish to estimate the rates of interaction (homophily) between socially salient groups.

Two main objectives of this project:

\begin{enumerate}
\item How does homophily in the sample relate to the social media platform?
\item How does homophily in the sample relate to the world?
\end{enumerate}

Answering 1 is doable. Answering 2 takes us into the realm of speculation and informed conjecture. While this

\subsection{Homophily calculation}

We follow the formalization in \cite{currarini_economic_2009}.

For a focal group $i$, call $s_i$ the average number of links to the ingroup and $d_i$ the average number of links to all outgroups.

Then we get a basic homophily measure

\begin{equation}
H_i = \frac{s_i}{d_i + s_i}
\end{equation}

This tells us nothing about random baselines, however. Group $i$ may be large or small. We denote $w_i$ the fraction of the population that belongs to $i$. If $H_i$ is higher than this baseline, we say that $i$ experiences \emph{inbreeding homophily}. We then normalize the statistic $H_i - w_i$ by the maximum possible bias, $1 - w_i$, which represents the case where there is perfect homophily (all ties ingroup minus the group size).

\begin{equation}
IH_i = \frac{H_i - w_i}{1 - w_i}
\end{equation}

\subsection{RDS framework}

We follow \cite{Salganik2004} and use the following notation. Assume that we have a graph with some nodes and edges. Further, assume all links are bidirectional. Then:

\begin{itemize}
\item $A$ and $B$ are socially relevant groups, such as race/ethnicity, gender, or educational attainment
\item $T_{AB}$: the number of links from $A$ to $B$
\item $C_{A,B}$: the probability an edge in $A$ leads to $B$
\item $R_A$: the total degree of $A$
\item $N_A$: the number of nodes in $A$
\item $D_A$: the average degree of $A$, $R_A / N_A$
\end{itemize}

Then we get the fundamental equality:

\begin{equation}
N_A D_A C_{A,B} = N_B D_B C_{B,A}
\end{equation}

This is true by definition assuming links are bidirected. It also gives us information about which information we need to infer the other parts. For instance, knowing any two of the following will let us infer the third: 1) number of nodes in each group; 2) the average degree of each group; 3) the probability of a crosslink.



\subsection{Random assignment of node characteristics}

\subsection{Incorporating homophily on characteristics}


\section{The effect of random node misclassification on homophily}

terms

\begin{enumerate}
\item $n_a$, $n_b$: number in $a$ and $b$
\item $N$: population size
\item $d_a$, $d_b$: number of links from a member of $a$, $b$
\item $d_{ab}$, $d_{ba}$: number of crossgroup links for each group, $a\to b$, $b\to a$
\item $d_a/n_a$, $d_b/n_b$: number of links per node
\item $c_{ab}$, $c_{ba}$: probability of crossgroup link from $a\to b$ and $b\to a$
\end{enumerate}

in expectation, there are $n_a * d_a * c_{ab}$ links from $a\to b$

misclassification messes up a few things:

1. estimates of population size
2. within-between link counts for each group
3. cross group link probability

say we mistake a random $a$ node for a $b$ node. this changes the group proportion estimates by $n_a/N - (n_a - 1)/N$ and $n_b/N + (n_b + 1)/N$.

it changes the crossgroup link probability by taking a random $a$ node and making its $a\to a$ links $b\to a$ and its $a\to b$ links $b\to b$. so we add the expected number of $d_{aa}$ links per node to $b\to a$ and the expected number of $d_{ab}$ links per node to $b\to b$

note that we can write $c_{ab} = d_{ab} / d_a$

then we get expressions for the updated $c_{aa}$, $c_{ab}$, $c_{ba}$, $c_{bb}$

\begin{enumerate}
\item $c_{aa} = (d_{aa} - E[d_{aa}/n_a]) / (d_a - E[d_{aa}/n_a])$ subtract expected $aa$ links
\item $c_{ab} = (d_{ab} - E[d_{ab}/n_a]) / (d_a - E[d_{ab}/n_a])$ subtract expected $ab$ links
\item $c_{ba} = (d_{ba} + E[d_{aa}/n_a]) / (d_b + E[d_{aa}/n_a])$ add expected $aa$ links
\item $c_{bb} = (d_{bb} + E[d_{ab}/n_a]) / (d_b + E[d_{ab}/n_a])$ add expected $ab$ links
\end{enumerate}


so we can see the expected effects of moving a random node from $a$ to $b$. since $d_a > d_{aa}$ (except in case of perfect homophily), subtracting the same number from the numerator and denominator of $d_{aa} / d_a$ will lead to an *underestimate* estimate of the true value. e.g. consider $(d_{aa} - x) / (d_a - x)$.

likewise we should get an over-estimate for $c_{ba}$, $c_{bb}$ since we basically have $(d_{ab} + x) / (d_b + x)$ and $(d_{bb} + x) / (d_b + x)$.

this is true for moving one node, but i think this argument fails for switching lots of nodes. for instance if we randomly flipped 20\% of the nodes in the graph, in expectation according to this argument we should not change homophily. but we clearly will by making the overall state of the graph "more random" and less ordered by homophily. the reason is that after switching lots of nodes the $d_{ab}$ and $d_{ba}$ numbers are substantially closer to random mixing than they were at the start.

how do we quanitify this?

we could make a dyadic independence assumption and see how well that works IRL

we really want like an estimate of how close to randomly mixed the links are after a certain number of random flips

Assume that each node has an equal probability p of being flipped. For each edge and node type, we can write out the possible transitions and their probabilities. For edges:

\begin{itemize}
\item The probability that the edge will remain as-is is $(1-p)^2$
\item The probability that only the first value of the edge will change is $p(1-p)$
\item The probability that only the second value of the edge will change is $p(1-p)$
\item The probability that both values of the edge will change is $p^2$
\end{itemize}

The total odds for any starting edge state are $(1-p)^2 + 2p(1-p) + p^2 = 1$.

And for nodes, the probability that the node will remain as is is $1-p$ and the probability that it will flip is $p$, by definition.

Then for each final edge- or node-state, the probability of that state is the sum of the products of the probabilities of all the initial states with their transition probability to the final state in question. For example, the probability that a link in the final network is from "a" to "a" is:

\begin{equation*}
c_{aa}' = (1-p)^2c_{aa} + p(1-p)c_{ab} + p(1-p)c_{ba} + p^2c_{bb}
\end{equation*}

And the probability that a node in the final network is "a" is:
\begin{equation*}
w_a' = (1-p)w_a + pw_b
\end{equation*}

The full systems of linear equations can be expressed in the form of a linear equation
\begin{equation*}
\vec{d}' = M\vec{d}
\end{equation*}
where for nodes we have
\begin{equation*}
\vec{d} = \begin{pmatrix}w_a\\w_b\end{pmatrix}
\end{equation*}
and
\begin{equation}
M = \begin{bmatrix}
1-p & p\\
p & 1-p
\end{bmatrix}
\end{equation}
and for edges we have
\begin{equation*}
\vec{d} = \begin{pmatrix}c_{aa}\\c_{ab}\\c_{ba}\\c_{bb}\end{pmatrix}
\end{equation*}
and
\begin{equation} M =
\begin{bmatrix}
(1-p)^2 & p(1-p) & p(1-p) & p^2\\
p(1-p) & (1-p)^2 & p^2 & p(1-p)\\
p(1-p) & p^2 & (1-p)^2 & p(1-p)\\
p^2 & p(1-p) & p(1-p) & (1-p)^2
\end{bmatrix}
\end{equation}

From this information, the desired metrics such as homophily can be calculated.

For different $p_a$ and $p_b$
\begin{equation}M =
\begin{bmatrix}
(1-p_a)^2 & p_b(1-p_a) & p_b(1-p_a) & p_bp_b\\
p_a(1-p_a) & (1-p_a)(1-p_b) & p_ap_b & p_b(1-p_b)\\
p_a(1-p_a) & p_ap_b & (1-p_a)(1-p_b) & p_b(1-p_b)\\
p_a^2 & p_a(1-p_b) & p_a(1-p_b) & (1-p_b)^2
\end{bmatrix}
\end{equation}
for edges and
\begin{equation}
M = \begin{bmatrix}
1-p_a & p_b\\
p_a & 1-p_b
\end{bmatrix}
\end{equation}
for nodes.

If you want to find the expected true edge and node counts after misclassification, then you can just invert this system.
\bibliographystyle{abbrv}
\bibliography{scibib}

\end{document}
